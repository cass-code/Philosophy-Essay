\documentclass[11pt,preprint, authoryear]{elsarticle}

\makeatletter
\renewcommand\@biblabel[1]{}
\makeatother

\usepackage{lmodern}
%%%% My spacing
\usepackage{setspace}
\setstretch{1.2}
\DeclareMathSizes{12}{14}{10}{10}

% Wrap around which gives all figures included the [H] command, or places it "here". This can be tedious to code in Rmarkdown.
\usepackage{float}
\let\origfigure\figure
\let\endorigfigure\endfigure
\renewenvironment{figure}[1][2] {
    \expandafter\origfigure\expandafter[H]
} {
    \endorigfigure
}

\let\origtable\table
\let\endorigtable\endtable
\renewenvironment{table}[1][2] {
    \expandafter\origtable\expandafter[H]
} {
    \endorigtable
}


\usepackage{ifxetex,ifluatex}
\usepackage{fixltx2e} % provides \textsubscript
\ifnum 0\ifxetex 1\fi\ifluatex 1\fi=0 % if pdftex
  \usepackage[T1]{fontenc}
  \usepackage[utf8]{inputenc}
\else % if luatex or xelatex
  \ifxetex
    \usepackage{mathspec}
    \usepackage{xltxtra,xunicode}
  \else
    \usepackage{fontspec}
  \fi
  \defaultfontfeatures{Mapping=tex-text,Scale=MatchLowercase}
  \newcommand{\euro}{€}
\fi

\usepackage{amssymb, amsmath, amsthm, amsfonts}

\def\bibsection{\section*{References}} %%% Make "References" appear before bibliography


\usepackage[round]{natbib}

\usepackage{longtable}
\usepackage[margin=2.3cm,bottom=2cm,top=2.5cm, includefoot]{geometry}
\usepackage{fancyhdr}
\usepackage[bottom, hang, flushmargin]{footmisc}
\usepackage{graphicx}
\numberwithin{equation}{section}
\numberwithin{figure}{section}
\numberwithin{table}{section}
\setlength{\parindent}{0cm}
\setlength{\parskip}{1.3ex plus 0.5ex minus 0.3ex}
\usepackage{textcomp}
\renewcommand{\headrulewidth}{0.2pt}
\renewcommand{\footrulewidth}{0.3pt}

\usepackage{array}
\newcolumntype{x}[1]{>{\centering\arraybackslash\hspace{0pt}}p{#1}}

%%%%  Remove the "preprint submitted to" part. Don't worry about this either, it just looks better without it:
\makeatletter
\def\ps@pprintTitle{%
  \let\@oddhead\@empty
  \let\@evenhead\@empty
  \let\@oddfoot\@empty
  \let\@evenfoot\@oddfoot
}
\makeatother

 \def\tightlist{} % This allows for subbullets!

\usepackage{hyperref}
\hypersetup{breaklinks=true,
            bookmarks=true,
            colorlinks=true,
            citecolor=blue,
            urlcolor=blue,
            linkcolor=blue,
            pdfborder={0 0 0}}


% The following packages allow huxtable to work:
\usepackage{siunitx}
\usepackage{multirow}
\usepackage{hhline}
\usepackage{calc}
\usepackage{tabularx}
\usepackage{booktabs}
\usepackage{caption}


\newenvironment{columns}[1][]{}{}

\newenvironment{column}[1]{\begin{minipage}{#1}\ignorespaces}{%
\end{minipage}
\ifhmode\unskip\fi
\aftergroup\useignorespacesandallpars}

\def\useignorespacesandallpars#1\ignorespaces\fi{%
#1\fi\ignorespacesandallpars}

\makeatletter
\def\ignorespacesandallpars{%
  \@ifnextchar\par
    {\expandafter\ignorespacesandallpars\@gobble}%
    {}%
}
\makeatother

\newlength{\cslhangindent}
\setlength{\cslhangindent}{1.5em}
\newenvironment{CSLReferences}%
  {\setlength{\parindent}{0pt}%
  \everypar{\setlength{\hangindent}{\cslhangindent}}\ignorespaces}%
  {\par}


\urlstyle{same}  % don't use monospace font for urls
\setlength{\parindent}{0pt}
\setlength{\parskip}{6pt plus 2pt minus 1pt}
\setlength{\emergencystretch}{3em}  % prevent overfull lines
\setcounter{secnumdepth}{5}

%%% Use protect on footnotes to avoid problems with footnotes in titles
\let\rmarkdownfootnote\footnote%
\def\footnote{\protect\rmarkdownfootnote}
\IfFileExists{upquote.sty}{\usepackage{upquote}}{}

%%% Include extra packages specified by user

%%% Hard setting column skips for reports - this ensures greater consistency and control over the length settings in the document.
%% page layout
%% paragraphs
\setlength{\baselineskip}{12pt plus 0pt minus 0pt}
\setlength{\parskip}{12pt plus 0pt minus 0pt}
\setlength{\parindent}{0pt plus 0pt minus 0pt}
%% floats
\setlength{\floatsep}{12pt plus 0 pt minus 0pt}
\setlength{\textfloatsep}{20pt plus 0pt minus 0pt}
\setlength{\intextsep}{14pt plus 0pt minus 0pt}
\setlength{\dbltextfloatsep}{20pt plus 0pt minus 0pt}
\setlength{\dblfloatsep}{14pt plus 0pt minus 0pt}
%% maths
\setlength{\abovedisplayskip}{12pt plus 0pt minus 0pt}
\setlength{\belowdisplayskip}{12pt plus 0pt minus 0pt}
%% lists
\setlength{\topsep}{10pt plus 0pt minus 0pt}
\setlength{\partopsep}{3pt plus 0pt minus 0pt}
\setlength{\itemsep}{5pt plus 0pt minus 0pt}
\setlength{\labelsep}{8mm plus 0mm minus 0mm}
\setlength{\parsep}{\the\parskip}
\setlength{\listparindent}{\the\parindent}
%% verbatim
\setlength{\fboxsep}{5pt plus 0pt minus 0pt}



\begin{document}



%titlepage
\thispagestyle{empty}
\begin{center}
\begin{minipage}{1\linewidth}
    \centering
%Entry1
    {\uppercase{\huge An Exploration of Human Rationality with Empirical
Evidence\par}}
    \vspace{0.9cm}
%Author's name
    {\LARGE \textbf{Cassandra Pengelly}\par}
    \vspace{0.9cm}
%Supervisor's Details
    {\LARGE \textbf{20346212}\par}
    \vspace{0.9cm}
%University logo
\begin{center}
    \includegraphics[width=0.4\linewidth]{img/statue.png}
\end{center}
\vspace{0.9cm}
%Degree
    {\LARGE Philosophy and Economics 871\par}
    \vspace{0.9cm}
%Institution
    {\LARGE 22 October 2021\par}
    \vspace{0.9cm}
%Date
    {\LARGE Word Count: 2980}
    \vspace{0.9cm}
%More
    {\normalsize }
%More
    {\normalsize }
\end{minipage}
\end{center}
\clearpage


\begin{frontmatter}  %

\title{}

% Set to FALSE if wanting to remove title (for submission)


\vspace{1cm}





\vspace{0.5cm}

\end{frontmatter}


\renewcommand{\contentsname}{Table of Contents}
{\tableofcontents}

%________________________
% Header and Footers
%%%%%%%%%%%%%%%%%%%%%%%%%%%%%%%%%
\pagestyle{fancy}
\chead{}
\rhead{}
\lfoot{}
\rfoot{\footnotesize Page \thepage}
\lhead{}
%\rfoot{\footnotesize Page \thepage } % "e.g. Page 2"
\cfoot{}

%\setlength\headheight{30pt}
%%%%%%%%%%%%%%%%%%%%%%%%%%%%%%%%%
%________________________

\headsep 35pt % So that header does not go over title




\newpage

\hypertarget{introduction}{%
\section{\texorpdfstring{Introduction
\label{Introduction}}{Introduction }}\label{introduction}}

Are Humans Rational? • Yes -- Anderson, Chater • No -- Tversky,
Kahneman, (Voltaire) • They do pretty well with limited resources --
Simon, Gigerenzer

When describing what it means to be human, famous Greek philosopher
Aristotle defined human beings as rational animals. In fact, Aristotle
emphasised that it is \emph{rationality} that makes humans unique and
distinct from animals (\protect\hyperlink{ref-aristotle}{Kietzmann,
2019}). But exactly what it means for people to be rational has long
been an interdisciplinary debate, drawing arguments from fields
including philosophy, economics, psychology and mathematics. Nobel
prizes

This essay\footnote{This essay was written in R using the package
  Texevier by \protect\hyperlink{ref-Texevier}{Katzke}
  (\protect\hyperlink{ref-Texevier}{2017}) and the write up can be found
  on Github \href{https://github.com/cass-code/Phil_essay}{here}}
explores what it means to be rational and whether humans fit some
definition of rationality. This essay is organised as follows. Section
\ref{def} presents different frameworks for defining rationality,
including rational choice theory and bounded rationality. Section
\ref{rev} briefly compares different views on human rationality in the
literature. Section \ref{case} presents three case studies that provide
empirical evidence on rational decision-making. And the final section
(\ref{con}) concludes.

\hypertarget{defining-rationality}{%
\section{\texorpdfstring{Defining Rationality
\label{def}}{Defining Rationality }}\label{defining-rationality}}

In order to understand if humans are rational, we first have to build a
framework to define what we mean by rationality. Definitions of
rationality relate to Max Weber's principle of methodological
individualism\footnote{The methodological precept that social phenomena
  should be explained as the result of individual
  actions(\protect\hyperlink{ref-weber}{Weber, 1922})}, since theories
of rational actions underlie theories of rationality. It follows then
that theories of rationality are classically linked to the primacy of
``the action frame of reference''
(\protect\hyperlink{ref-types}{Demeulenaere, 2014: 517};
\protect\hyperlink{ref-parsons}{Parsons, 1937: 43--51}). The social
sciences have defined rationality in a number of different ways; the
five most general notions of rationality.

All of them can be related, more or less clearly, to Popper's
``problem-solving'' notion (Popper, 1967).

\hypertarget{rational-choice-theory}{%
\subsection{Rational Choice Theory}\label{rational-choice-theory}}

\hypertarget{bounded-rationality}{%
\subsection{Bounded Rationality}\label{bounded-rationality}}

Neoclassical economics assumes that people are perfectly rational,
whereas behavioural economics uses psychology and economic theory to
create more realistic models of human decision-making
(\protect\hyperlink{ref-rabin}{Rabin, 2002}). People are subject to
certain biases and often make use of heuristics in their decision-making
process, which can lead to predictable errors in judgment
\protect\hyperlink{ref-prospect}{Kahneman \& Tversky}
(\protect\hyperlink{ref-prospect}{1979}). Behavioural economics
literature investigates how these biases can be combated to improve
welfare outcomes. \protect\hyperlink{ref-nudge}{Thaler \& Sunstein}
(\protect\hyperlink{ref-nudge}{2008}) introduced the idea of a
nudge\footnote{Nudge: an intervention that alters behaviour towards a
  desired action. In order for an intervention to qualify as a nudge, it
  should be cheap and easy to avoid
  (\protect\hyperlink{ref-nudge}{Thaler \& Sunstein, 2008}).} as a way
to guide people to make better choices. For example, changing the
default option for organ donation to be opt-in as opposed to explicit
consent could benefit potential donors (who were deterred by the
registration process) and save more lives
(\protect\hyperlink{ref-nudge}{Thaler \& Sunstein, 2008: 176}).

The concepts of loss aversion, reference dependence and regret avoidance
can also be included in health interventions through a ``regret
lottery''. \protect\hyperlink{ref-prospect}{Kahneman \& Tversky}
(\protect\hyperlink{ref-prospect}{1979}) describe loss aversion as a
cognitive bias whereby people experience losses as more painful than the
pleasure they receive from an equivalent gain. Thus, people are more
willing to take on risk to avoid a loss, and are less risk-seeking when
pursuing gain (\protect\hyperlink{ref-prospect}{Kahneman \& Tversky,
1979: 268}). Reference dependence follows on from loss aversion and
suggests that people define gains and losses relative to a reference
point (\protect\hyperlink{ref-ref}{Tversky \& Kahneman, 1991: 1039}).
People are also subject to regret avoidance, where there is a
significant emotional cost attached to regret and people make decisions
to avoid regretting alternative decisions in the future
(\protect\hyperlink{ref-regret}{Bailey \& Kinerson, 2005}).

\hypertarget{the-great-human-rationality-debate}{%
\section{\texorpdfstring{The Great Human Rationality Debate
\label{rev}}{The Great Human Rationality Debate }}\label{the-great-human-rationality-debate}}

\hypertarget{case-studies}{%
\section{\texorpdfstring{Case Studies
\label{case}}{Case Studies }}\label{case-studies}}

\hypertarget{conclusion}{%
\section{\texorpdfstring{Conclusion
\label{con}}{Conclusion }}\label{conclusion}}

The behavioural literature and empirical studies show that lotteries can
be an effective method to incentivise vaccine take-up, and South
Africans appear to be well-primed for such a health intervention. This
field experiment is designed to test this hypothesis.

\newpage

\hypertarget{references}{%
\section*{References}\label{references}}
\addcontentsline{toc}{section}{References}

\hypertarget{refs}{}
\begin{CSLReferences}{1}{0}
\leavevmode\hypertarget{ref-regret}{}%
Bailey, J.J. \& Kinerson, C. 2005. Regret avoidance and risk tolerance.
\emph{Journal of Financial Counseling and Planning}. 16(1):23.

\leavevmode\hypertarget{ref-types}{}%
Demeulenaere, P. 2014. Are there many types of rationality.
\emph{Papers. Revista de Sociologia}. 99(4):515--528.

\leavevmode\hypertarget{ref-fast}{}%
Kahneman, D. 2011. \emph{Thinking, fast and slow}. Macmillan.

\leavevmode\hypertarget{ref-prospect}{}%
Kahneman, D. \& Tversky, A. 1979. Prospect theory: An analysis of
decision under risk. \emph{Econometrica}. 47(2):263--291.

\leavevmode\hypertarget{ref-Texevier}{}%
Katzke, N.F. 2017. \emph{{Texevier}: {P}ackage to create elsevier
templates for rmarkdown}. Stellenbosch, South Africa: Bureau for
Economic Research.

\leavevmode\hypertarget{ref-aristotle}{}%
Kietzmann, C. 2019. \emph{Aristotle on the definition of what it is to
be human}. G. Keil \& N. Kreft (eds.). Cambridge University Press.

\leavevmode\hypertarget{ref-parsons}{}%
Parsons, T. 1937. \emph{The structure of social action}. New York: Free
Press.

\leavevmode\hypertarget{ref-rabin}{}%
Rabin, M. 2002. A perspective on psychology and economics.
\emph{European economic review}. 46(4-5):657--685.

\leavevmode\hypertarget{ref-nudge}{}%
Thaler, R. \& Sunstein, C. 2008. \emph{Nudge: Improving decisions about
health, wealth, and happiness.} New Haven, CT: Yale University Press.

\leavevmode\hypertarget{ref-khan}{}%
Tversky, A. \& Kahneman, D. 1974. Judgment under uncertainty: Heuristics
and biases. \emph{Science}. 185(4157):1124--1131.

\leavevmode\hypertarget{ref-ref}{}%
Tversky, A. \& Kahneman, D. 1991. Loss aversion in riskless choice: A
reference-dependent model. \emph{The Quarterly Journal of Economics}.
106(4):1039--1061. {[}Online{]}, Available:
\url{http://www.jstor.org/stable/2937956}.

\leavevmode\hypertarget{ref-weber}{}%
Weber, M. 1922. \emph{Economy and society}. Guenther Roth \& W. Claus
(eds.). Berkeley: University of California Press, 1968.

\end{CSLReferences}

\bibliography{Tex/ref}





\end{document}
