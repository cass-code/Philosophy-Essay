\documentclass[11pt,preprint, authoryear]{elsarticle}

\makeatletter
\renewcommand\@biblabel[1]{}
\makeatother

\usepackage{lmodern}
%%%% My spacing
\usepackage{setspace}
\setstretch{1.2}
\DeclareMathSizes{12}{14}{10}{10}

% Wrap around which gives all figures included the [H] command, or places it "here". This can be tedious to code in Rmarkdown.
\usepackage{float}
\let\origfigure\figure
\let\endorigfigure\endfigure
\renewenvironment{figure}[1][2] {
    \expandafter\origfigure\expandafter[H]
} {
    \endorigfigure
}

\let\origtable\table
\let\endorigtable\endtable
\renewenvironment{table}[1][2] {
    \expandafter\origtable\expandafter[H]
} {
    \endorigtable
}


\usepackage{ifxetex,ifluatex}
\usepackage{fixltx2e} % provides \textsubscript
\ifnum 0\ifxetex 1\fi\ifluatex 1\fi=0 % if pdftex
  \usepackage[T1]{fontenc}
  \usepackage[utf8]{inputenc}
\else % if luatex or xelatex
  \ifxetex
    \usepackage{mathspec}
    \usepackage{xltxtra,xunicode}
  \else
    \usepackage{fontspec}
  \fi
  \defaultfontfeatures{Mapping=tex-text,Scale=MatchLowercase}
  \newcommand{\euro}{€}
\fi

\usepackage{amssymb, amsmath, amsthm, amsfonts}

\def\bibsection{\section*{References}} %%% Make "References" appear before bibliography


\usepackage[round]{natbib}

\usepackage{longtable}
\usepackage[margin=2.3cm,bottom=2cm,top=2.5cm, includefoot]{geometry}
\usepackage{fancyhdr}
\usepackage[bottom, hang, flushmargin]{footmisc}
\usepackage{graphicx}
\numberwithin{equation}{section}
\numberwithin{figure}{section}
\numberwithin{table}{section}
\setlength{\parindent}{0cm}
\setlength{\parskip}{1.3ex plus 0.5ex minus 0.3ex}
\usepackage{textcomp}
\renewcommand{\headrulewidth}{0.2pt}
\renewcommand{\footrulewidth}{0.3pt}

\usepackage{array}
\newcolumntype{x}[1]{>{\centering\arraybackslash\hspace{0pt}}p{#1}}

%%%%  Remove the "preprint submitted to" part. Don't worry about this either, it just looks better without it:
\makeatletter
\def\ps@pprintTitle{%
  \let\@oddhead\@empty
  \let\@evenhead\@empty
  \let\@oddfoot\@empty
  \let\@evenfoot\@oddfoot
}
\makeatother

 \def\tightlist{} % This allows for subbullets!

\usepackage{hyperref}
\hypersetup{breaklinks=true,
            bookmarks=true,
            colorlinks=true,
            citecolor=blue,
            urlcolor=blue,
            linkcolor=blue,
            pdfborder={0 0 0}}


% The following packages allow huxtable to work:
\usepackage{siunitx}
\usepackage{multirow}
\usepackage{hhline}
\usepackage{calc}
\usepackage{tabularx}
\usepackage{booktabs}
\usepackage{caption}


\newenvironment{columns}[1][]{}{}

\newenvironment{column}[1]{\begin{minipage}{#1}\ignorespaces}{%
\end{minipage}
\ifhmode\unskip\fi
\aftergroup\useignorespacesandallpars}

\def\useignorespacesandallpars#1\ignorespaces\fi{%
#1\fi\ignorespacesandallpars}

\makeatletter
\def\ignorespacesandallpars{%
  \@ifnextchar\par
    {\expandafter\ignorespacesandallpars\@gobble}%
    {}%
}
\makeatother

\newlength{\cslhangindent}
\setlength{\cslhangindent}{1.5em}
\newenvironment{CSLReferences}%
  {\setlength{\parindent}{0pt}%
  \everypar{\setlength{\hangindent}{\cslhangindent}}\ignorespaces}%
  {\par}


\urlstyle{same}  % don't use monospace font for urls
\setlength{\parindent}{0pt}
\setlength{\parskip}{6pt plus 2pt minus 1pt}
\setlength{\emergencystretch}{3em}  % prevent overfull lines
\setcounter{secnumdepth}{5}

%%% Use protect on footnotes to avoid problems with footnotes in titles
\let\rmarkdownfootnote\footnote%
\def\footnote{\protect\rmarkdownfootnote}
\IfFileExists{upquote.sty}{\usepackage{upquote}}{}

%%% Include extra packages specified by user

%%% Hard setting column skips for reports - this ensures greater consistency and control over the length settings in the document.
%% page layout
%% paragraphs
\setlength{\baselineskip}{12pt plus 0pt minus 0pt}
\setlength{\parskip}{12pt plus 0pt minus 0pt}
\setlength{\parindent}{0pt plus 0pt minus 0pt}
%% floats
\setlength{\floatsep}{12pt plus 0 pt minus 0pt}
\setlength{\textfloatsep}{20pt plus 0pt minus 0pt}
\setlength{\intextsep}{14pt plus 0pt minus 0pt}
\setlength{\dbltextfloatsep}{20pt plus 0pt minus 0pt}
\setlength{\dblfloatsep}{14pt plus 0pt minus 0pt}
%% maths
\setlength{\abovedisplayskip}{12pt plus 0pt minus 0pt}
\setlength{\belowdisplayskip}{12pt plus 0pt minus 0pt}
%% lists
\setlength{\topsep}{10pt plus 0pt minus 0pt}
\setlength{\partopsep}{3pt plus 0pt minus 0pt}
\setlength{\itemsep}{5pt plus 0pt minus 0pt}
\setlength{\labelsep}{8mm plus 0mm minus 0mm}
\setlength{\parsep}{\the\parskip}
\setlength{\listparindent}{\the\parindent}
%% verbatim
\setlength{\fboxsep}{5pt plus 0pt minus 0pt}



\begin{document}



%titlepage
\thispagestyle{empty}
\begin{center}
\begin{minipage}{1\linewidth}
    \centering
%Entry1
    {\uppercase{\huge An Exploration of Human Rationality with Empirical
Evidence\par}}
    \vspace{0.9cm}
%Author's name
    {\LARGE \textbf{Cassandra Pengelly}\par}
    \vspace{0.9cm}
%Supervisor's Details
    {\LARGE \textbf{20346212}\par}
    \vspace{0.9cm}
%University logo
\begin{center}
    \includegraphics[width=0.4\linewidth]{img/statue.png}
\end{center}
\vspace{0.9cm}
%Degree
    {\LARGE Philosophy and Economics 871\par}
    \vspace{0.9cm}
%Institution
    {\LARGE 22 October 2021\par}
    \vspace{0.9cm}
%Date
    {\LARGE Word Count: 2980}
    \vspace{0.9cm}
%More
    {\normalsize }
%More
    {\normalsize }
\end{minipage}
\end{center}
\clearpage


\begin{frontmatter}  %

\title{}

% Set to FALSE if wanting to remove title (for submission)


\vspace{1cm}





\vspace{0.5cm}

\end{frontmatter}


\renewcommand{\contentsname}{Table of Contents}
{\tableofcontents}

%________________________
% Header and Footers
%%%%%%%%%%%%%%%%%%%%%%%%%%%%%%%%%
\pagestyle{fancy}
\chead{}
\rhead{}
\lfoot{}
\rfoot{\footnotesize Page \thepage}
\lhead{}
%\rfoot{\footnotesize Page \thepage } % "e.g. Page 2"
\cfoot{}

%\setlength\headheight{30pt}
%%%%%%%%%%%%%%%%%%%%%%%%%%%%%%%%%
%________________________

\headsep 35pt % So that header does not go over title




\newpage

\hypertarget{introduction}{%
\section{\texorpdfstring{Introduction
\label{Introduction}}{Introduction }}\label{introduction}}

Are Humans Rational? • Yes -- Anderson, Chater • No -- Tversky,
Kahneman, (Voltaire) • They do pretty well with limited resources --
Simon, Gigerenzer

When describing what it means to be human, famous Greek philosopher
Aristotle defined human beings as rational animals. In fact, Aristotle
emphasised that it is \emph{rationality} that makes humans unique and
distinct from animals (\protect\hyperlink{ref-aristotle}{Kietzmann,
2019}). But exactly what it means for people to be rational has long
been an interdisciplinary debate, drawing arguments from fields
including philosophy, economics, psychology and mathematics. Nobel
prizes

This essay\footnote{This essay was written in R using the package
  Texevier by \protect\hyperlink{ref-Texevier}{Katzke}
  (\protect\hyperlink{ref-Texevier}{2017}) and the write up can be found
  on Github \href{https://github.com/cass-code/Phil_essay}{here}.}
explores what it means to be rational and whether humans fit some
definition of rationality. This essay is organised as follows. Section
\ref{def} presents five different criteria for defining rationality.
Section \ref{rev} briefly compares different views on human rationality
in the literature. Section \ref{case} presents three case studies that
provide empirical evidence on rational decision-making. And the final
section (\ref{con}) concludes.

\hypertarget{defining-rationality}{%
\section{\texorpdfstring{Defining Rationality
\label{def}}{Defining Rationality }}\label{defining-rationality}}

In order to understand if humans are rational, we first have to build a
framework to define what we mean by rationality. Definitions of
rationality relate to Max Weber's principle of methodological
individualism\^{}{[}The methodological precept that social phenomena
should be explained as the result of individual actions
(\protect\hyperlink{ref-weber}{Weber, 1922}), since theories of rational
actions underlie theories of rationality. It follows then that theories
of rationality are classically linked to the primacy of ``the action
frame of reference'' (\protect\hyperlink{ref-types}{Demeulenaere, 2014:
517}; \protect\hyperlink{ref-parsons}{Parsons, 1937: 43--51}). While the
social sciences have defined rationality in a number of different ways,
this section focuses on five of the most general notions of rationality.

The first notion regards an individual as rational if she acts in a way
that is intentional. This stems from Weber's sociology, where the
intention of the act is primary, and the outcome of the action is of
secondary importance (\protect\hyperlink{ref-fry}{Fry \& Raadschelders,
2013: 26}). A simple example of a rational act under this framework
would be for a person to eat ice-cream because she enjoys eating
ice-cream, or alternatively, not to eat ice-cream because she wants to
reduce her sugar intake. To act irrationally according to this theory
would be to pursue an action based on non-intentional causes
(\protect\hyperlink{ref-types}{Demeulenaere, 2014: 518}). For instance,
a lecturer intends to grade all his students' tests fairly, but he tends
to give higher marks to papers he grades after he's eaten
lunch\footnote{The lecturer's hunger may lead him to mark more strictly}.
In this case the lecturer is unaware of the influence of external
factors on his deliberate decision. Thus, intentional motives are not
fully governing the individual choice. It may also be that a person acts
in opposition to her intentions, such as eating ice-cream because she
craves sugar even though she would prefer not to eat ice-cream. While
intentionality is not a particularly strong criteria by which to judge
rationality, it is a useful starting point.

A second criterion for rationality is that choices should be transitive.
As posited by \protect\hyperlink{ref-sen}{Sen}
(\protect\hyperlink{ref-sen}{1977: 323}), neoclassical economics defines
a rational person as one who is internally consistent in how he orders
his subjective preferences. For example, if a person prefers ice-cream
to chocolate, and he prefers chocolate to apples, it must follow that he
prefers ice-cream to apples. There is no restriction or structure placed
on the decision itself -- there is no ``correct'' decision. However, if
a person made inconsistent decisions, he would be considered irrational.
For example, a person choosing to eat apples when there is ice-cream
available, given that he prefers ice-cream to apples. As shown by
\protect\hyperlink{ref-math}{Smith, Eggen \& Andre}
(\protect\hyperlink{ref-math}{2014: 147}), transitivity can be written
in mathematical terms as:

\begin{align*}
Let A be a set and R be a relation on A \newline
R is transitive iff for all x, y, and z} %in% A, if xRy and yRz, then xRz.\newline
For the ice-cream example above, whenever x > y and y > z, then also x > z
\end{align*}\footnote{Where \(x=\)ice-cream, \(y\)=chocolate and
  \(z\)=apples}

This approach of definitional egoism sometimes goes under the name of
rational choice, and it involves nothing other than internal
consistency. A person's choices are considered ``rational'' in this ap-
proach if and only if these choices can all be explained in terms of
some preference relation consistent with the revealed preference defi-
nition, that is, if all his choices can be explained as the choosing of
``most preferred'' alternatives with respect to a postulated preference
relation.'2 The rationale of this approach seems to be based on the idea
that the only way of understanding a person's real preference is to
examine his actual choices, and there is no choice-independent way of
understanding someone's attitude towards alternative

The second notion of rationality is linked to the ideal norm of
consistency or transitivity of choices. The rationality does not stem
from the contents of individual choices as such, but just from the fact
that individuals are consistent in the ordering of their subjective
preferences. Preferences are clearly not considered to be rational in
this case. It is the official position of neo-classical economics, where
rationality does not involve any kind of ``right'' decision, but only
the fact that people do not take inconsistent decisions (Sen, 1977).
Irrationality here would correspond to inconsistent choices of the kind,
again, described by Kahneman and Tversky (2000), or Elster (2010). Here
again,

The first one is to consider that individuals are rational whenever they
act in an intentional way. The intentional decision to act corresponds
to the reason someone has to act, for instance smoking for her pleasure,
or not smoking in order to avoid health problems. This classical idea
can be found in Weber.

An irrational behaviour would be then to act on the basis of
non-intentional causes. For instance, Kahneman (2011) reports a study
about parole judges observed in their decisions. In this example,
individual judges tend to be more severe when they are hungrier: clearly
in this case non-intentional and unconscious causes affect intentional
decisions. The judges are not aware of the influence of those factors on
their deliberate decisions. This means that intentional motives do not
have the sole influence on individual decisions; however, in this
example it is still intentional decisions that are at stake, partly
determined by unconscious trends. It can be also the case that people
will act in a manner that is opposed to their intentions, for instance
when they smoke although they would prefer not to smoke. Intentional
action can clearly be linked

All of them can be related, more or less clearly, to Popper's
``problem-solving'' notion (Popper, 1967).

\hypertarget{rational-choice-theory}{%
\subsection{Rational Choice Theory}\label{rational-choice-theory}}

\hypertarget{bounded-rationality}{%
\subsection{Bounded Rationality}\label{bounded-rationality}}

Neoclassical economics assumes that people are perfectly rational,
whereas behavioural economics uses psychology and economic theory to
create more realistic models of human decision-making
(\protect\hyperlink{ref-rabin}{Rabin, 2002}). People are subject to
certain biases and often make use of heuristics in their decision-making
process, which can lead to predictable errors in judgment
\protect\hyperlink{ref-prospect}{Kahneman \& Tversky}
(\protect\hyperlink{ref-prospect}{1979}). Behavioural economics
literature investigates how these biases can be combated to improve
welfare outcomes. \protect\hyperlink{ref-nudge}{Thaler \& Sunstein}
(\protect\hyperlink{ref-nudge}{2008}) introduced the idea of a
nudge\footnote{Nudge: an intervention that alters behaviour towards a
  desired action. In order for an intervention to qualify as a nudge, it
  should be cheap and easy to avoid
  (\protect\hyperlink{ref-nudge}{Thaler \& Sunstein, 2008}).} as a way
to guide people to make better choices. For example, changing the
default option for organ donation to be opt-in as opposed to explicit
consent could benefit potential donors (who were deterred by the
registration process) and save more lives
(\protect\hyperlink{ref-nudge}{Thaler \& Sunstein, 2008: 176}).

The concepts of loss aversion, reference dependence and regret avoidance
can also be included in health interventions through a ``regret
lottery''. \protect\hyperlink{ref-prospect}{Kahneman \& Tversky}
(\protect\hyperlink{ref-prospect}{1979}) describe loss aversion as a
cognitive bias whereby people experience losses as more painful than the
pleasure they receive from an equivalent gain. Thus, people are more
willing to take on risk to avoid a loss, and are less risk-seeking when
pursuing gain (\protect\hyperlink{ref-prospect}{Kahneman \& Tversky,
1979: 268}). Reference dependence follows on from loss aversion and
suggests that people define gains and losses relative to a reference
point (\protect\hyperlink{ref-ref}{Tversky \& Kahneman, 1991: 1039}).
People are also subject to regret avoidance, where there is a
significant emotional cost attached to regret and people make decisions
to avoid regretting alternative decisions in the future
(\protect\hyperlink{ref-regret}{Bailey \& Kinerson, 2005}).

\hypertarget{the-great-debate}{%
\section{\texorpdfstring{The Great Debate
\label{rev}}{The Great Debate }}\label{the-great-debate}}

\hypertarget{case-studies}{%
\section{\texorpdfstring{Case Studies
\label{case}}{Case Studies }}\label{case-studies}}

\hypertarget{conclusion}{%
\section{\texorpdfstring{Conclusion
\label{con}}{Conclusion }}\label{conclusion}}

One of the benefits of exploring human rationality is the insight we
gain into human behaviour and decision-making. Getting a clearer . The
behavioural literature and empirical studies show that lotteries can be
an effective method to incentivise vaccine take-up, and South Africans
appear to be well-primed for such a health intervention. This field
experiment is designed to test this hypothesis.

\newpage

\hypertarget{references}{%
\section*{References}\label{references}}
\addcontentsline{toc}{section}{References}

\hypertarget{refs}{}
\begin{CSLReferences}{1}{0}
\leavevmode\hypertarget{ref-regret}{}%
Bailey, J.J. \& Kinerson, C. 2005. Regret avoidance and risk tolerance.
\emph{Journal of Financial Counseling and Planning}. 16(1):23.

\leavevmode\hypertarget{ref-types}{}%
Demeulenaere, P. 2014. Are there many types of rationality.
\emph{Papers. Revista de Sociologia}. 99(4):515--528.

\leavevmode\hypertarget{ref-fry}{}%
Fry, B.R. \& Raadschelders, J.C. 2013. \emph{Mastering public
administration: From max weber to dwight waldo}. CQ Press.

\leavevmode\hypertarget{ref-fast}{}%
Kahneman, D. 2011. \emph{Thinking, fast and slow}. Macmillan.

\leavevmode\hypertarget{ref-prospect}{}%
Kahneman, D. \& Tversky, A. 1979. Prospect theory: An analysis of
decision under risk. \emph{Econometrica}. 47(2):263--291.

\leavevmode\hypertarget{ref-Texevier}{}%
Katzke, N.F. 2017. \emph{{Texevier}: {P}ackage to create elsevier
templates for rmarkdown}. Stellenbosch, South Africa: Bureau for
Economic Research.

\leavevmode\hypertarget{ref-aristotle}{}%
Kietzmann, C. 2019. \emph{Aristotle on the definition of what it is to
be human}. G. Keil \& N. Kreft (eds.). Cambridge University Press.

\leavevmode\hypertarget{ref-parsons}{}%
Parsons, T. 1937. \emph{The structure of social action}. New York: Free
Press.

\leavevmode\hypertarget{ref-rabin}{}%
Rabin, M. 2002. A perspective on psychology and economics.
\emph{European economic review}. 46(4-5):657--685.

\leavevmode\hypertarget{ref-sen}{}%
Sen, A.K. 1977. Rational fools: A critique of the behavioral foundations
of economic theory. \emph{Philosophy \& Public Affairs}. 6(4):317--344.
{[}Online{]}, Available: \url{http://www.jstor.org/stable/2264946}.

\leavevmode\hypertarget{ref-math}{}%
Smith, D., Eggen, M. \& Andre, R.S. 2014. \emph{A transition to advanced
mathematics}. Cengage Learning.

\leavevmode\hypertarget{ref-nudge}{}%
Thaler, R. \& Sunstein, C. 2008. \emph{Nudge: Improving decisions about
health, wealth, and happiness.} New Haven, CT: Yale University Press.

\leavevmode\hypertarget{ref-khan}{}%
Tversky, A. \& Kahneman, D. 1974. Judgment under uncertainty: Heuristics
and biases. \emph{Science}. 185(4157):1124--1131.

\leavevmode\hypertarget{ref-ref}{}%
Tversky, A. \& Kahneman, D. 1991. Loss aversion in riskless choice: A
reference-dependent model. \emph{The Quarterly Journal of Economics}.
106(4):1039--1061. {[}Online{]}, Available:
\url{http://www.jstor.org/stable/2937956}.

\leavevmode\hypertarget{ref-weber}{}%
Weber, M. 1922. \emph{Economy and society}. Guenther Roth \& W. Claus
(eds.). Berkeley: University of California Press, 1968.

\end{CSLReferences}

\bibliography{Tex/ref}





\end{document}
