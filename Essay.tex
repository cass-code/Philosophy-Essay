\documentclass[11pt,preprint, authoryear]{elsarticle}

\makeatletter
\renewcommand\@biblabel[1]{}
\makeatother

\usepackage{lmodern}
%%%% My spacing
\usepackage{setspace}
\setstretch{1.2}
\DeclareMathSizes{12}{14}{10}{10}

% Wrap around which gives all figures included the [H] command, or places it "here". This can be tedious to code in Rmarkdown.
\usepackage{float}
\let\origfigure\figure
\let\endorigfigure\endfigure
\renewenvironment{figure}[1][2] {
    \expandafter\origfigure\expandafter[H]
} {
    \endorigfigure
}

\let\origtable\table
\let\endorigtable\endtable
\renewenvironment{table}[1][2] {
    \expandafter\origtable\expandafter[H]
} {
    \endorigtable
}


\usepackage{ifxetex,ifluatex}
\usepackage{fixltx2e} % provides \textsubscript
\ifnum 0\ifxetex 1\fi\ifluatex 1\fi=0 % if pdftex
  \usepackage[T1]{fontenc}
  \usepackage[utf8]{inputenc}
\else % if luatex or xelatex
  \ifxetex
    \usepackage{mathspec}
    \usepackage{xltxtra,xunicode}
  \else
    \usepackage{fontspec}
  \fi
  \defaultfontfeatures{Mapping=tex-text,Scale=MatchLowercase}
  \newcommand{\euro}{€}
\fi

\usepackage{amssymb, amsmath, amsthm, amsfonts}

\def\bibsection{\section*{References}} %%% Make "References" appear before bibliography


\usepackage[round]{natbib}

\usepackage{longtable}
\usepackage[margin=2.3cm,bottom=2cm,top=2.5cm, includefoot]{geometry}
\usepackage{fancyhdr}
\usepackage[bottom, hang, flushmargin]{footmisc}
\usepackage{graphicx}
\numberwithin{equation}{section}
\numberwithin{figure}{section}
\numberwithin{table}{section}
\setlength{\parindent}{0cm}
\setlength{\parskip}{1.3ex plus 0.5ex minus 0.3ex}
\usepackage{textcomp}
\renewcommand{\headrulewidth}{0.2pt}
\renewcommand{\footrulewidth}{0.3pt}

\usepackage{array}
\newcolumntype{x}[1]{>{\centering\arraybackslash\hspace{0pt}}p{#1}}

%%%%  Remove the "preprint submitted to" part. Don't worry about this either, it just looks better without it:
\makeatletter
\def\ps@pprintTitle{%
  \let\@oddhead\@empty
  \let\@evenhead\@empty
  \let\@oddfoot\@empty
  \let\@evenfoot\@oddfoot
}
\makeatother

 \def\tightlist{} % This allows for subbullets!

\usepackage{hyperref}
\hypersetup{breaklinks=true,
            bookmarks=true,
            colorlinks=true,
            citecolor=blue,
            urlcolor=blue,
            linkcolor=blue,
            pdfborder={0 0 0}}


% The following packages allow huxtable to work:
\usepackage{siunitx}
\usepackage{multirow}
\usepackage{hhline}
\usepackage{calc}
\usepackage{tabularx}
\usepackage{booktabs}
\usepackage{caption}


\newenvironment{columns}[1][]{}{}

\newenvironment{column}[1]{\begin{minipage}{#1}\ignorespaces}{%
\end{minipage}
\ifhmode\unskip\fi
\aftergroup\useignorespacesandallpars}

\def\useignorespacesandallpars#1\ignorespaces\fi{%
#1\fi\ignorespacesandallpars}

\makeatletter
\def\ignorespacesandallpars{%
  \@ifnextchar\par
    {\expandafter\ignorespacesandallpars\@gobble}%
    {}%
}
\makeatother

\newlength{\cslhangindent}
\setlength{\cslhangindent}{1.5em}
\newenvironment{CSLReferences}%
  {\setlength{\parindent}{0pt}%
  \everypar{\setlength{\hangindent}{\cslhangindent}}\ignorespaces}%
  {\par}


\urlstyle{same}  % don't use monospace font for urls
\setlength{\parindent}{0pt}
\setlength{\parskip}{6pt plus 2pt minus 1pt}
\setlength{\emergencystretch}{3em}  % prevent overfull lines
\setcounter{secnumdepth}{5}

%%% Use protect on footnotes to avoid problems with footnotes in titles
\let\rmarkdownfootnote\footnote%
\def\footnote{\protect\rmarkdownfootnote}
\IfFileExists{upquote.sty}{\usepackage{upquote}}{}

%%% Include extra packages specified by user

%%% Hard setting column skips for reports - this ensures greater consistency and control over the length settings in the document.
%% page layout
%% paragraphs
\setlength{\baselineskip}{12pt plus 0pt minus 0pt}
\setlength{\parskip}{12pt plus 0pt minus 0pt}
\setlength{\parindent}{0pt plus 0pt minus 0pt}
%% floats
\setlength{\floatsep}{12pt plus 0 pt minus 0pt}
\setlength{\textfloatsep}{20pt plus 0pt minus 0pt}
\setlength{\intextsep}{14pt plus 0pt minus 0pt}
\setlength{\dbltextfloatsep}{20pt plus 0pt minus 0pt}
\setlength{\dblfloatsep}{14pt plus 0pt minus 0pt}
%% maths
\setlength{\abovedisplayskip}{12pt plus 0pt minus 0pt}
\setlength{\belowdisplayskip}{12pt plus 0pt minus 0pt}
%% lists
\setlength{\topsep}{10pt plus 0pt minus 0pt}
\setlength{\partopsep}{3pt plus 0pt minus 0pt}
\setlength{\itemsep}{5pt plus 0pt minus 0pt}
\setlength{\labelsep}{8mm plus 0mm minus 0mm}
\setlength{\parsep}{\the\parskip}
\setlength{\listparindent}{\the\parindent}
%% verbatim
\setlength{\fboxsep}{5pt plus 0pt minus 0pt}



\begin{document}



%titlepage
\thispagestyle{empty}
\begin{center}
\begin{minipage}{1\linewidth}
    \centering
%Entry1
    {\uppercase{\huge An Exploration of Human Rationality\par}}
    \vspace{0.9cm}
%Author's name
    {\LARGE \textbf{Cassandra Pengelly}\par}
    \vspace{0.9cm}
%Supervisor's Details
    {\LARGE \textbf{20346212}\par}
    \vspace{0.9cm}
%University logo
\begin{center}
    \includegraphics[width=0.4\linewidth]{img/statue.png}
\end{center}
\vspace{0.9cm}
%Degree
    {\LARGE Philosophy and Economics 871\par}
    \vspace{0.9cm}
%Institution
    {\LARGE 22 October 2021\par}
    \vspace{0.9cm}
%Date
    {\LARGE Word Count: 2980}
    \vspace{0.9cm}
%More
    {\normalsize }
%More
    {\normalsize }
\end{minipage}
\end{center}
\clearpage


\begin{frontmatter}  %

\title{}

% Set to FALSE if wanting to remove title (for submission)


\vspace{1cm}





\vspace{0.5cm}

\end{frontmatter}


\renewcommand{\contentsname}{Table of Contents}
{\tableofcontents}

%________________________
% Header and Footers
%%%%%%%%%%%%%%%%%%%%%%%%%%%%%%%%%
\pagestyle{fancy}
\chead{}
\rhead{}
\lfoot{}
\rfoot{\footnotesize Page \thepage}
\lhead{}
%\rfoot{\footnotesize Page \thepage } % "e.g. Page 2"
\cfoot{}

%\setlength\headheight{30pt}
%%%%%%%%%%%%%%%%%%%%%%%%%%%%%%%%%
%________________________

\headsep 35pt % So that header does not go over title




\newpage

\hypertarget{introduction}{%
\section{\texorpdfstring{Introduction
\label{Introduction}}{Introduction }}\label{introduction}}

When describing what it means to be human, famous Greek philosopher
Aristotle defined human beings as rational animals. In fact, Aristotle
emphasised that it is \emph{rationality} that makes humans unique and
distinct from animals (\protect\hyperlink{ref-aristotle}{Kietzmann,
2019}). But exactly what it means for people to be rational has long
been an interdisciplinary debate, drawing arguments from fields
including philosophy, economics, psychology and mathematics. The
importance of

This essay\footnote{This essay was written in R using the package
  Texevier by \protect\hyperlink{ref-Texevier}{Katzke}
  (\protect\hyperlink{ref-Texevier}{2017}) and the write up can be found
  on Github \href{https://github.com/cass-code/Phil_essay}{here}.}
explores what it means to be rational and investigates whether we can
classify humans as rational beings. This essay is organised as follows.
Section \ref{def} presents different criteria and theories for defining
rationality. Section \ref{rev} briefly compares different views on human
rationality in the literature. Section \ref{case} presents three case
studies that provide empirical evidence on rational decision-making
regarding gym membership contracts, the stock market and litigation. And
the final section (\ref{con}) concludes.

\hypertarget{defining-rationality}{%
\section{\texorpdfstring{Defining Rationality
\label{def}}{Defining Rationality }}\label{defining-rationality}}

In order to understand if humans are rational, we first have to build a
framework to define what we mean by rationality. Definitions of
rationality relate to Max Weber's principle of methodological
individualism\footnote{The methodological precept that social phenomena
  should be explained as the result of individual actions
  (\protect\hyperlink{ref-weber}{Weber, 1922})}, since theories of
rational actions underlie theories of rationality. It follows then that
theories of rationality are classically linked to the primacy of ``the
action frame of reference'' (\protect\hyperlink{ref-types}{Demeulenaere,
2014: 517}; \protect\hyperlink{ref-parsons}{Parsons, 1937: 43--51}).
While the social sciences have defined rationality in a number of
different ways, this section focuses on five of the most general notions
of rationality.

The first notion regards an individual as rational if she acts in a way
that is intentional. This stems from Weber's sociology, where the
intention of the act is primary, and the outcome of the action is of
secondary importance (\protect\hyperlink{ref-fry}{Fry \& Raadschelders,
2013: 26}). A simple example of a rational act under this framework
would be for a person to eat ice-cream because she enjoys eating
ice-cream, or alternatively, not to eat ice-cream because she wants to
reduce her sugar intake. To act irrationally according to this theory
would be to pursue an action based on non-intentional causes
(\protect\hyperlink{ref-types}{Demeulenaere, 2014: 518}). For instance,
a lecturer intends to grade all his students' tests fairly, but he tends
to give higher marks to papers he grades after he's eaten
lunch\footnote{The lecturer's hunger may lead him to unintentionally
  mark more strictly.}. In this case the lecturer is unaware of the
influence of external factors on his deliberate decision. Thus,
intentional motives are not fully governing the individual choice. It
may also be that a person acts in opposition to her intentions, such as
eating ice-cream because she craves sugar even though she would prefer
not to eat ice-cream. While intentionality is not a particularly strong
criterion by which to judge rationality, it is a useful starting point.

The second criterion for rationality is that choices should be
transitive. This approach defines a rational person as one who is
internally consistent in how he orders his subjective preferences
(\protect\hyperlink{ref-sen}{Sen, 1977: 323}). For example, if a person
prefers ice-cream to chocolate, and he prefers chocolate to apples, it
must follow that he prefers ice-cream to apples. There is no restriction
or structure placed on the decision itself -- there is no ``correct''
decision. However, if a person makes inconsistent decisions, he would be
considered irrational. For example, a person choosing to eat apples when
there is ice-cream available, given that he prefers ice-cream to apples.
The theory supporting this criterion is that observing a person's actual
choices is the only way of understanding an individual's real
preferences. As shown by \protect\hyperlink{ref-math}{Smith, Eggen \&
Andre} (\protect\hyperlink{ref-math}{2014: 147}), transitivity can be
written in mathematical terms as:

\(Let \; A \; be \;a\; set \;and\; R\; be \;a \;relation \;on\; A:\)
\newline
\(R\; is \;transitive \;iff \;for \;all \;x, \;y,\; and \; z \in A, \;if\; xRy \;and\; yRz,\; then\; xRz.\)

Applying this definition to the ice-cream example above, whenever
\(x > y\) and \(y > z\), then also \(x > z\), where \(x=\) ice-cream,
\(y=\) chocolate and \(z=\) apples.

The third theory of rationality defines a rational individual as one who
chooses the sufficient means to achieve an end. This instrumental
rationality implies the normative notion of correctness; rationality is
dependent on the ability of a person to select the ``correct'' means to
reach an end (\protect\hyperlink{ref-types}{Demeulenaere, 2014}). For
example, it would be instrumentally rational to start a business if an
individual has an end goal of making profit.

The fourth criterion of rationality is that individuals pursue their
self-interest efficiently. This concept of rationality is more strict as
it imposes structure on both the choice of an end and the means to
achieve this end, as well as requiring consistency of the ends. This
approach is related to rational choice theory and utility maximisation,
which are widely used in neoclassical economics. Under utilitarianism,
which has its roots in hedomism\footnote{The philosopical theory that
  people are motivated in life by pleasure and pain
  \protect\hyperlink{ref-hed}{Moore} (\protect\hyperlink{ref-hed}{2019})}

A similar concept to rational choice theory is that of bounded
rationality, as proposed by
(\protect\hyperlink{ref-simon}{\textbf{simon?}})

The fifth notion of rationality defines a rational person as one that
has a good reason, from her point of view, for her actions. The fact
that a person has \emph{good} reasons and not just any reasons is the
main idea behind Boudon's theory of rationality
(\protect\hyperlink{ref-boudon}{Boudon, 1989}). The theory presupposes
that there is an ideally good choice that is attainable, but a person
could still be rational even if she does not make this ``ideal'' choice.
Rather, a rational individual makes a reasonable choice from her
perspective, given her circumstances and available information. This is
similar to bounded rationality but differs in that Boudon explicitly
emphasises good reasons.

All of them can be related, more or less clearly, to Popper's
``problem-solving'' notion (Popper, 1967).

Neoclassical economics assumes that people are perfectly rational,
whereas behavioural economics uses psychology and economic theory to
create more realistic models of human decision-making
(\protect\hyperlink{ref-rabin}{Rabin, 2002}). People are subject to
certain biases and often make use of heuristics in their decision-making
process, which can lead to predictable errors in judgment
\protect\hyperlink{ref-prospect}{Kahneman \& Tversky}
(\protect\hyperlink{ref-prospect}{1979}). Behavioural economics
literature investigates how these biases can be combated to improve
welfare outcomes. \protect\hyperlink{ref-nudge}{Thaler \& Sunstein}
(\protect\hyperlink{ref-nudge}{2008}) introduced the idea of a
nudge\footnote{Nudge: an intervention that alters behaviour towards a
  desired action. In order for an intervention to qualify as a nudge, it
  should be cheap and easy to avoid
  (\protect\hyperlink{ref-nudge}{Thaler \& Sunstein, 2008}).} as a way
to guide people to make better choices. For example, changing the
default option for organ donation to be opt-in as opposed to explicit
consent could benefit potential donors (who were deterred by the
registration process) and save more lives
(\protect\hyperlink{ref-nudge}{Thaler \& Sunstein, 2008: 176}).

The concepts of loss aversion, reference dependence and regret avoidance
can also be included in health interventions through a ``regret
lottery''. \protect\hyperlink{ref-prospect}{Kahneman \& Tversky}
(\protect\hyperlink{ref-prospect}{1979}) describe loss aversion as a
cognitive bias whereby people experience losses as more painful than the
pleasure they receive from an equivalent gain. Thus, people are more
willing to take on risk to avoid a loss, and are less risk-seeking when
pursuing gain (\protect\hyperlink{ref-prospect}{Kahneman \& Tversky,
1979: 268}). Reference dependence follows on from loss aversion and
suggests that people define gains and losses relative to a reference
point (\protect\hyperlink{ref-ref}{Tversky \& Kahneman, 1991: 1039}).
People are also subject to regret avoidance, where there is a
significant emotional cost attached to regret and people make decisions
to avoid regretting alternative decisions in the future
(\protect\hyperlink{ref-regret}{Bailey \& Kinerson, 2005}).

\hypertarget{the-great-debate}{%
\section{\texorpdfstring{The Great Debate
\label{rev}}{The Great Debate }}\label{the-great-debate}}

Are Humans Rational? • Yes -- Anderson, Chater • No -- Tversky,
Kahneman, (Voltaire) • They do pretty well with limited resources --
Simon, Gigerenzer

\hypertarget{case-studies}{%
\section{\texorpdfstring{Case Studies
\label{case}}{Case Studies }}\label{case-studies}}

\hypertarget{gym-membership-486}{%
\subsection{Gym Membership 486}\label{gym-membership-486}}

A standard assumption in classical economic literature is that agents
hold rational expectations regarding their future consumption and make
choices on a utility-maximising basis.
\protect\hyperlink{ref-gym}{DellaVigna \& Malmendier}
(\protect\hyperlink{ref-gym}{2006}) test this assumption by
investigating how people choose gym contracts. The authors use a data
set from three American health clubs, which includes information on the
types of contracts that members hold, the cost of these contracts, and
the daily attendance of 7,752 club members for three years. First,
\protect\hyperlink{ref-gym}{DellaVigna \& Malmendier}
(\protect\hyperlink{ref-gym}{2006}) construct a contract choice model
with the assumption that going to the gym has an immediate effort cost
and future health benefits; the model also assumes that the cost of
effort is unknown ex ante. Additionally, the model assumes customers pay
a fee to exercise and can choose from and flat-fee, monthly or annual
contracts The authors then make 3 different predictions based on
rational beliefs and standard preferences. The first prediction is the
price per expected attendance. The second prediction is that members who
took out monthly contracts would have, on average, a lower initial
attendance than those who took out annual contracts. And the final
prediction is that the average actual gym attendance equals the average
forecast of attendance.

The paper finds that people do not choose their optimal contract, given
how often they go to the gym (\protect\hyperlink{ref-gym}{DellaVigna \&
Malmendier, 2006: 716}). The empirical analysis shows that 80\% of
monthly members would have paid less under a pay-per-visit scheme for
the same number of visits. And members who took out monthly contracts
upwards of 70\$ would have paid 70\% less under the flat-fee system for
the same number of visits. A rational utility maximiser would opt to pay
less for a given number of visits if he derived positive utility from
having money. Thus, paying more for a given number of gym sessions
suggests that these individuals are not rational. Monthly gym
subscribers can cancel their membership in any month but are charged a
higher fee per month than the annual contract. Individuals who are more
unsure of how often they will gym each month would prefer the monthly
contract because it's more flexible. And those who are more sure that
they will consistently gym would prefer annual memberships because they
work out to be cheaper. Therefore under a rational system of beliefs,
annual members have a higher likelihood of being frequent club users
than monthly ones and renewing their annual gym membership. The data
shows that the second prediction does not holds and that annual members
are 17\% less likely to remain enrolled for longer than a year than
monthly subscribers.

Finally, the third prediction is contradicted by the results, where the
average forecasted monthly visits was 9.5 and the average actual
attendance was 4.17 visits per month. The authors conclude that their
results are not in line with the predictions of a rational model, and
that people likely overestimate their future efficiency or future
self-control (\protect\hyperlink{ref-gym}{DellaVigna \& Malmendier,
2006: 716}). In a related study, \protect\hyperlink{ref-gymm}{Acland \&
Levy} (\protect\hyperlink{ref-gymm}{2015}) ran an experiment to
investigate gym attendance and found that their test subjects tended to
greatly overpredict the number of future gym visits. The paper concluded
from the results that their subjects were presently biased and naive
about self-control problems. These results and interpretation are
supported by \protect\hyperlink{ref-comm}{Carrera, Royer, Stehr, Sydnor
\& Taubinsky} (\protect\hyperlink{ref-comm}{2019: 4}) who analysed
commitment preferences and found that participants held overoptimistic
beliefs about gym attendance and could be classified as naive hyperbolic
discounters.

\hypertarget{investing}{%
\subsection{Investing}\label{investing}}

\hypertarget{litigation}{%
\subsection{Litigation}\label{litigation}}

\hypertarget{conclusion}{%
\section{\texorpdfstring{Conclusion
\label{con}}{Conclusion }}\label{conclusion}}

One of the benefits of exploring human rationality is the insight we
gain into human behaviour and decision-making. Getting a clearer . The
behavioural literature and empirical studies show that lotteries can be
an effective method to incentivise vaccine take-up, and South Africans
appear to be well-primed for such a health intervention. This field
experiment is designed to test this hypothesis.

\newpage

\hypertarget{references}{%
\section*{References}\label{references}}
\addcontentsline{toc}{section}{References}

\hypertarget{refs}{}
\begin{CSLReferences}{1}{0}
\leavevmode\hypertarget{ref-gymm}{}%
Acland, D. \& Levy, M. 2015. Naiveté, projection bias, and habit
formation in gym attendance. \emph{Management Science}. 61(1):146--160.
{[}Online{]}, Available: \url{http://www.jstor.org/stable/24551076}.

\leavevmode\hypertarget{ref-regret}{}%
Bailey, J.J. \& Kinerson, C. 2005. Regret avoidance and risk tolerance.
\emph{Journal of Financial Counseling and Planning}. 16(1):23.

\leavevmode\hypertarget{ref-boudon}{}%
Boudon, R. 1989. Subjective rationality and the explanation of social
behavior. \emph{Rationality and society}. 1(2):173--196.

\leavevmode\hypertarget{ref-comm}{}%
Carrera, M., Royer, H., Stehr, M., Sydnor, J.R. \& Taubinsky, D. 2019.
\emph{Who chooses commitment? Evidence and welfare implications}. NBER
Working Paper No. w26161. {[}Online{]}, Available:
\url{https://ssrn.com/abstract=3439171}.

\leavevmode\hypertarget{ref-gym}{}%
DellaVigna, S. \& Malmendier, U. 2006. Paying not to go to the gym.
\emph{American Economic Review}. 96:694--719.

\leavevmode\hypertarget{ref-types}{}%
Demeulenaere, P. 2014. Are there many types of rationality.
\emph{Papers. Revista de Sociologia}. 99(4):515--528.

\leavevmode\hypertarget{ref-fry}{}%
Fry, B.R. \& Raadschelders, J.C. 2013. \emph{Mastering public
administration: From max weber to dwight waldo}. CQ Press.

\leavevmode\hypertarget{ref-fast}{}%
Kahneman, D. 2011. \emph{Thinking, fast and slow}. Macmillan.

\leavevmode\hypertarget{ref-prospect}{}%
Kahneman, D. \& Tversky, A. 1979. Prospect theory: An analysis of
decision under risk. \emph{Econometrica}. 47(2):263--291.

\leavevmode\hypertarget{ref-Texevier}{}%
Katzke, N.F. 2017. \emph{{Texevier}: {P}ackage to create elsevier
templates for rmarkdown}. Stellenbosch, South Africa: Bureau for
Economic Research.

\leavevmode\hypertarget{ref-aristotle}{}%
Kietzmann, C. 2019. \emph{Aristotle on the definition of what it is to
be human}. G. Keil \& N. Kreft (eds.). Cambridge University Press.

\leavevmode\hypertarget{ref-hed}{}%
Moore, A. 2019. Hedonism: Subjective rationality and the explanation of
social behavior. \emph{The Stanford Encyclopedia of Philosophy (Winter
2019 Edition)}. {[}Online{]}, Available:
\url{https://plato.stanford.edu/archives/win2019/entries/hedonism/}.

\leavevmode\hypertarget{ref-parsons}{}%
Parsons, T. 1937. \emph{The structure of social action}. New York: Free
Press.

\leavevmode\hypertarget{ref-rabin}{}%
Rabin, M. 2002. A perspective on psychology and economics.
\emph{European economic review}. 46(4-5):657--685.

\leavevmode\hypertarget{ref-sen}{}%
Sen, A.K. 1977. Rational fools: A critique of the behavioral foundations
of economic theory. \emph{Philosophy \& Public Affairs}. 6(4):317--344.
{[}Online{]}, Available: \url{http://www.jstor.org/stable/2264946}.

\leavevmode\hypertarget{ref-math}{}%
Smith, D., Eggen, M. \& Andre, R.S. 2014. \emph{A transition to advanced
mathematics}. Cengage Learning.

\leavevmode\hypertarget{ref-nudge}{}%
Thaler, R. \& Sunstein, C. 2008. \emph{Nudge: Improving decisions about
health, wealth, and happiness.} New Haven, CT: Yale University Press.

\leavevmode\hypertarget{ref-khan}{}%
Tversky, A. \& Kahneman, D. 1974. Judgment under uncertainty: Heuristics
and biases. \emph{Science}. 185(4157):1124--1131.

\leavevmode\hypertarget{ref-ref}{}%
Tversky, A. \& Kahneman, D. 1991. Loss aversion in riskless choice: A
reference-dependent model. \emph{The Quarterly Journal of Economics}.
106(4):1039--1061. {[}Online{]}, Available:
\url{http://www.jstor.org/stable/2937956}.

\leavevmode\hypertarget{ref-weber}{}%
Weber, M. 1922. \emph{Economy and society}. Guenther Roth \& W. Claus
(eds.). Berkeley: University of California Press, 1968.

\end{CSLReferences}

\bibliography{Tex/ref}





\end{document}
